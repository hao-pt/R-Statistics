% !TEX encoding = UTF-8 Unicode
\documentclass[a4paper,12pt]{article}

%-----------------------------------------Include package & set up some thing-----------------------------------------------
\usepackage{fontspec}
\setmainfont{Times New Roman} %set font
\usepackage{enumitem} % to format list
\usepackage{amsmath}
\usepackage{listings} % quote code
\usepackage{color}
\usepackage{hyperref} % cite hyperlink & bookmarks
\usepackage{setspace} % space

\hypersetup{unicode, colorlinks,linkcolor=black, urlcolor=cyan} % format hyperlink and bookmarks

%Define title
\title{Báo cáo bài tập 1}
\author{1612174 - Phùng Tiến Hào - \href{mailto:tienhaophung@gmail.com}{tienhaophung@gmail.com}}
\date{19-03-2019}

%Code formatting with the listing package
\definecolor{codegreen}{rgb}{0,0.6,0}
\definecolor{codegray}{rgb}{0.3,0.3,0.3}
\definecolor{codepurple}{rgb}{0.58,0,0.82}
\definecolor{backcolour}{rgb}{0.92,0.92,0.88}

\lstdefinestyle{mystyle}{
	backgroundcolor=\color{backcolour},   
	commentstyle=\color{codegreen},
	keywordstyle=\color{blue},
	numberstyle=\tiny\color{codegray},
	stringstyle=\color{codepurple},
	basicstyle=\footnotesize,
	breakatwhitespace=false,         
	breaklines=true,                 
	captionpos=b,                    
	keepspaces=true,                 
	numbers=left,                    
	numbersep=5pt,                  
	showspaces=false,                
	showstringspaces=false,
	showtabs=false,                  
	tabsize=2
}

\lstset{style=mystyle}

\begin{document}
\pagenumbering{gobble}
\maketitle
\newpage

\doublespacing
\tableofcontents
\singlespace

\newpage
\pagenumbering{arabic}

\section{Câu 1}
\label{Cau 1}
(5đ). Xét thí nghiệm gieo xúc xắc (đồng chất) 2 lần, tính xác suất các biến cố: \\
\vspace{0.5cm}\\
$A =$ 'Số chấm ở lần I'\\
$B =$ 'Số chấm ở lần II'\\
$N(\omega) = 6 * 6 = 36$ \\
\begin{enumerate}[label=\alph*)]
	\item Tổng số chấm không quá 5 \\
	
	$C =$ 'Tổng số chấm không quá 5' \\
	\begin{align*}
		N(C) &= |\{(1, 1), (1, 2), (1, 3), (2, 1), (2, 2), (3, 1)\}| = 6 \\
		P(C) &= \frac{N(C)}{N(\omega)} = \frac{6}{36} = \frac{1}{6} \approx 0.167
	\end{align*}
	
	\item Số chấm của hai lần là như nhau \\
	\begin{align*}
		|A = B| &= |\{(1, 1), (2, 2), (3, 3), (4, 4), (5, 5), (6, 6)\}| = 6 \\
		P(A = B) &= \frac{|A = B|}{N(\omega)} = \frac{6}{36} = \frac{1}{6} \approx 0.167
	\end{align*}
	
	\item Số chấm của lần thứ nhất là chẵn nhưng số chấm của lần thứ hai là lẻ \\
	\begin{equation*}
		P((A \% 2 == 0) \cap (B \% 2 == 1)) = \frac{3*3}{36} = \frac{9}{36} = \frac{1}{4} = 0.25
	\end{equation*}
	
	\item Số chấm của lần thứ nhất lớn hơn lần thứ hai khi biết số chấm của lần thứ nhất không quá 4\\
	\begin{align*}
		N(A < 4) &= |{1, 2, 3}| = 3 \\
		N(A > B \cap A < 4) &= |\{(2, 1), (3, 1), (3, 2)\}| = 3 \\
		P(A<4) &= \frac{3}{6} \\
		P(A > B \cap A < 4) &= \frac{3}{36} \\
		P(A > B | A < 4) &= \frac{P(A > B \cap A < 4)}{P(A<4)} = \displaystyle{\frac{\frac{3}{6}}{\frac{3}{36}}} = \frac{1}{6} \approx 0.167
	\end{align*}
	
	\item Tổng số chấm lớn hơn 9 khi biết có ít nhất một lần được số chấm lớn hơn 3 \\
	\begin{align*}
		P(X > 3 + Y > 3) &= P(X > 3) + P(Y > 3) - P(X > 3, Y > 3) \\
		&= \frac{3}{6} + \frac{3}{6} - \frac{3}{6}*\frac{3}{6} = \frac{3}{4}
	\end{align*}
	
	\begin{align*}
		N(X+Y > 9, X > 3 + Y > 3) &= |{(4, 6), (5, 5), (5,6), (6,4), (6,5), (6,6)}| = 6 \\
		P(X+Y > 9, X > 3 + Y > 3) &= \frac{6}{36}
	\end{align*}
	
	\begin{align*}
		P(X + Y > 9 | X>3 + Y>3) &= \frac{P(X+Y > 9, X > 3 + Y > 3)}{P(X > 3 + Y > 3)} \\
		&= \displaystyle{\frac{\frac{6}{36}}{\frac{3}{4}}} = \frac{2}{9} \approx 0.222
	\end{align*}
	
\end{enumerate}

{\large\textbf{Mô phỏng trong R}} \\

\begin{itemize}
	\item Yêu cầu: a, b, c \\
	\begin{lstlisting}[language=R]
	tung_xuc_sac <- function(yeu_cau, count = 0){
		n <- 1:6
		xs <- sample(n, 2, replace = TRUE)
		if(yeu_cau == 1){ #a) Tong so cham khong qua 5
			return (sum(xs) < 5)
		}
		else if(yeu_cau == 2){ #b) So cham cua 2 lan nhu nhau
		return (xs[1] == xs[2])  
		}
		else if(yeu_cau == 3){ #c) So cham lan I la chan, lan II la le
		return (xs[1] %% 2 == 0 && xs[2] %% 2 != 0)
		}
		return (FALSE)
	}
	
	tan_suat <- function(n, yeu_cau){
		if(yeu_cau < 4){
			kq <- replicate(n, tung_xuc_sac(yeu_cau))
			return (sum(kq)/n)
		}
	}
	
	#Test
	#a) Tong so cham khong qua 5
		> tan_suat(100000, 1)
		[1] 0.16747
	#b) So cham cua 2 lan nhu nhau
		> tan_suat(300000, 2)
		[1] 0.1665533
	#c) So cham lan I la chan, lan II la le
		> tan_suat(50000, 3)
		[1] 0.24882
	\end{lstlisting}
	
	\item Yêu cầu: d \\
	\begin{lstlisting}[language=R]
		#d) So cham lan thu I > so cham lan II khi lan I < 4
		tn_tung_xuc_xac_co_dk <- function(n){
			count <- 0
			count1 <- 0
			for(i in 1:n){
				range <- 1:6
				xs <- sample(range, 2, replace = TRUE)
				if(xs[1] < 4){
					count <- count + 1
					if(xs[1] > xs[2])
						count1 <- count1 + 1
				}
			}
			return (count1/count)
		}
		#Test
		#d) So cham lan thu I > so cham lan II khi lan I < 4
			> tn_tung_xuc_xac_co_dk(100000)
			[1] 0.1641135
	\end{lstlisting}
	\item Yêu cầu: e\\
	\begin{lstlisting}[language=R]
	tung_xuc_sac_co_tong_so_cham_lon_hon_9 <- function(n){
		count <- 0
		count1 <- 0
		for(i in 1:n){
			range <- 1:6
			xs <- sample(range, 2, replace = TRUE)
			if(xs[1] > 3 || xs[2] > 3){
				count <- count + 1
				if(sum(xs) > 9)
					count1 <- count1 + 1
			}
		}
		return (count1/count)
	}
	
	#e) Tong so cham > 9 khi biet it nhat 1 lan duoc so cham > 3
		> tung_xuc_sac_co_tong_so_cham_lon_hon_9(5000)
		[1] 0.2172627
	\end{lstlisting}
	
\end{itemize}	
	
	
	\section{Câu 2}
	\label{Cau 2}
	(5 đ). Có 3 hộp đựng bi. Hộp I có 5 bi xanh và 4 bi đỏ. Hộp II có 8 bi xanh, 4 bi đỏ và 5 bi
	vàng. Hộp III có 6 bi xanh, 2 bi đỏ và 4 bi vàng. Chọn ngẫu nhiên một hộp rồi từ hộp đó lấy ngẫu
	nhiên một viên bi. \\
	\vspace{0.5cm}\\
	$A_1 = $ 'Lấy bi màu xanh' \\
	$A_2 = $ 'Lấy bi màu đỏ'\\
	$A_3 = $ 'Lấy bi màu vàng'\\
	$B_i = $ 'Lấy hộp thứ i' $i = 1, 2, 3$ \\
	
	\begin{enumerate}[label=\alph*)]
		\item Tìm xác suất để lấy được bi xanh. \\
		\begin{align*}
		P(B_i) &= \frac{1}{3} \vee i \\
		P(A_1) &= P(A_1|B_1)P(B_1) + P(A_1|B_2) P(B_2) + P(A_1|B_3)P(B_3) \\
		&= (\frac{5}{9} + \frac{8}{17} + \frac{6}{12}) * \frac{1}{3} = \frac{467}{918} \approx 0.509	
		\end{align*}
		
		\item Tìm xác suất để lấy được bi xanh của hộp I. \\
		\begin{align*}
		P(A_1 B_1) = P(A_1|B_1)P(B_1) = \frac{5}{9} * \frac{1}{3} = \frac{5}{27} \approx 0.185
		\end{align*}
		
		\item Nếu biết rằng bi lấy được là bi vàng, khả năng bi đó thuộc về hộp nào là lớn nhất? \\
		\begin{align*}
		P(A_3) &= P(A_3|B_1)P(B_1) + P(A_3|B_2)P(B_2) + P(A_3|B_3)P(B_3) \\
		&= \frac{1}{3} * \left (0 + \frac{5}{17} + \frac{4}{12} \right) = \frac{32}{153}
		\end{align*}
		
		\begin{align*}	
		P(B_1|A_3) &= 0 \\
		P(B_2|A_3) &= \frac{P(A_3|B_2)P(B_2)}{P(A_3)} = \displaystyle{\frac{\frac{5}{17} * \frac{1}{3}}{\frac{32}{153}}} = \frac{15}{32} \approx 0.469 \\
		P(B_3|A_3) &= \frac{P(A_3|B_3)P(B_3)}{P(A_3)} = \displaystyle{\frac{\frac{4}{12} * \frac{1}{3}}{\frac{32}{153}}} = \frac{17}{32} \approx 0.531
		\end{align*}
		
		-> Xác xuất bi vàng thuộc về hộp III là lớn nhất
		\end{enumerate}
	
	{\large\textbf{Mô phỏng trong R}} \\
	\begin{itemize}
		\item Yêu cầu a, b \\
	
		\begin{lstlisting}[language=R]
		lay_bi <- function(yeuCau){
			hop <- c(1,2,3)
			hop1 <- c(rep('X', 5), rep('D', 4))
			hop2 <- c(rep('X', 8), rep('D', 4), rep('V', 5))
			hop3 <- c(rep('X', 6), rep('D', 2), rep('V', 4))
			
			#Lay ngau nhien 1 hop
			idx_hop_chon <- sample(hop, 1)
			hop_chon <- c()
			if(idx_hop_chon == 1)
				hop_chon <- hop1
			else if(idx_hop_chon == 2)
				hop_chon <- hop2
			else
				hop_chon <- hop3
				
			#Lay ngau nhien 1 bi tu hop chon
			bi_chon <- sample(hop_chon, 1)
		
			if(yeuCau == 1){ #a) Lay duoc bi xanh
				return (bi_chon == 'X')
			}
			else if(yeuCau == 2){ #b) Lay duoc bi xanh cua hop I
				return (bi_chon == 'X' && idx_hop_chon == 1)
			}
			return (FALSE)
		}
		
		tan_suat_bi <- function(n, yeu_cau){
			kq <- replicate(n, lay_bi(yeu_cau))
			return (sum(kq)/n)
		}
		
		#Test
		#a) Lay duoc bi xanh
			> tan_suat_bi(50000, 1)
			[1] 0.51042
		#b) Lay duoc bi xanh cua hop I
			> tan_suat_bi(100000, 2)
			[1] 0.18581
		\end{lstlisting}
		
		\item Yêu cầu c \\
		\begin{lstlisting}[language=R]
			#c) Biet lay duoc bi vang, kha nang bi do thuoc ve hop nao lon nhat?
			lay_bi_vang <- function(n){
				hop <- c(1,2,3)
				hop1 <- c(rep('X', 5), rep('D', 4))
				hop2 <- c(rep('X', 8), rep('D', 4), rep('V', 5))
				hop3 <- c(rep('X', 6), rep('D', 2), rep('V', 4))
			
				count <- c(0,0,0) #Dem so TH co bi vang trong tung hop
				countBiVang <- 0 #Dem so TH co bi vang 
			
				for (i in 1:n){
					#Chon ngau nhien 1 hop
					idx_hop_chon <- sample(hop, 1)
					# Chon ngau nhien 1 vien bi tu hop da chon
					hop_chon <- c()
					if(idx_hop_chon == 1)
						hop_chon <- hop1
					else if(idx_hop_chon == 2)
						hop_chon <- hop2
					else
						hop_chon <- hop3
						bi_chon <- sample(hop_chon, 1)
			
					if(bi_chon == 'V'){
						countBiVang <- countBiVang + 1
					if(idx_hop_chon == 1)
						count[1] <- count[1] + 1
					else if(idx_hop_chon == 2)
						count[2] <- count[2] + 1
					else if(idx_hop_chon == 3)
						count[3] <- count[3] + 1
				}
			}
			
				#Tim bi vang thuoc ve hop co xs cao nhat
				idx_max <- 0
				maxCount <- max(count)
				if(maxCount == count[1])
					idx_max <- 1
				else if(maxCount == count[2])
					idx_max <- 2
				else if(maxCount == count[3])
					idx_max <- 3
			
				return (c(idx_max, maxCount / countBiVang))
			}
			
			#Test
			#c) Biet lay bi vang, kha nang bi vang thuoc ve hop nao lon nhat?
				> lay_bi_vang(50000)
				[1] 3.0000000 0.5285903
		\end{lstlisting}
		
	\end{itemize}

	\section{Tham khảo}
	\label{Tham khao}
	[1] Introduction to R, \href{https://campus.datacamp.com/courses/free-introduction-to-r}{Datacamp}. \\

		

\end{document}