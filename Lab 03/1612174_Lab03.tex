% !TEX encoding = UTF-8 Unicode
\documentclass[a4paper,12pt]{article}

%-----------------------------------------Include package & set up some thing-----------------------------------------------
\usepackage{fontspec}
\setmainfont{Times New Roman} %set font
\usepackage{enumitem} % to format list
\usepackage{amsmath}
\usepackage{listings} % quote code
\usepackage{color}
\usepackage{hyperref} % cite hyperlink & bookmarks
\usepackage{setspace} % space
\usepackage{graphicx} % insert image


\hypersetup{unicode, colorlinks,linkcolor=black, urlcolor=cyan} % format hyperlink and bookmarks

%Define title
\title{Báo cáo bài tập 3}
\author{1612174 - Phùng Tiến Hào - \href{mailto:tienhaophung@gmail.com}{tienhaophung@gmail.com}}
\date{29/03/2019}

%Code formatting with the listing package
\definecolor{codegreen}{rgb}{0,0.6,0}
\definecolor{codegray}{rgb}{0.3,0.3,0.3}
\definecolor{codepurple}{rgb}{0.58,0,0.82}
\definecolor{backcolour}{rgb}{0.92,0.92,0.88}

\lstdefinestyle{mystyle}{
	backgroundcolor=\color{backcolour},   
	commentstyle=\color{codegreen},
	keywordstyle=\color{blue},
	numberstyle=\tiny\color{codegray},
	stringstyle=\color{codepurple},
	basicstyle=\footnotesize,
	breakatwhitespace=false,         
	breaklines=true,                 
	captionpos=b,                    
	keepspaces=true,                 
	numbers=left,                    
	numbersep=5pt,                  
	showspaces=false,                
	showstringspaces=false,
	showtabs=false,                  
	tabsize=2
}

\lstset{style=mystyle}

\begin{document}
	\pagenumbering{gobble}
	\maketitle
	\newpage
	
	\doublespacing
	\tableofcontents
	\singlespace
	
	\newpage
	\pagenumbering{arabic}
	
	\section{Câu 1}
	(4 đ). Có 3 xạ thủ cùng bắn đạn vào bia. Xác suất để các xạ thủ bắn trúng bia lần lượt là
	0.2, 0.4, 0.6. Giả sử rằng mỗi xạ thủ chỉ bắn 1 viên đạn và việc bắn đạn của một xạ thủ không bị
	ảnh hưởng bởi các xạ thủ khác. Tính kì vọng của số đạn trúng bia.\\
	
	\begin{center}
		\textbf{Giải}
	\end{center}
	
	\begin{enumerate}[label = \alph*)]
		\item Tìm phân phối của số đạn trúng bia.\\
		
		Gọi $p_1, p_2, p_3$ lần lượt là xác xuất bắn trúng bia của ba thợ săn:
		
		\begin{equation*}
		\begin{cases}
		p_1 = 0.2 -> p_1^c = 0.8\\
		p_2 = 0.4 -> p_2^c = 0.6\\
		p_3 = 0.6 -> p_3^c = 0.4\\
		\end{cases}
		\end{equation*}
		
		\begin{align*}
		P(X = 0) &= p_1^c p_2^c p_3^c = 0.192\\
		P(X = 1) &= p_1 p_2^c p_3^c + p_1^c p_2 p_3^c + p_1^c p_2^c p_3 = 0.464\\
		P(X = 2) &= p_1 p_2 p_3^c + p_1^c p_2 p_3 + p_1 p_2^c p_3 = 0.296\\
		P(X = 3) &= p_1 p_2 p_3 = 0.048
		\end{align*}
		
		\begin{table}[h!]
			\begin{center}
				\caption{Bảng phân phối xác suất của X (với X là số viên đạn trung bia):}
				\begin{tabular}{|c|c|c|c|c|}
					\hline 
					X & 0 & 1 & 2 & 3 \\ 
					\hline 
					P(X = x) & 0.192 & 0.464 & 0.296 & 0.048 \\ 
					\hline 
				\end{tabular} 
			\end{center}
		\end{table}
		
		\item Tính kì vọng của số đạn trúng bia.
		
		\begin{equation*}
			E(X) = \Sigma p_ix_i = 1.2
		\end{equation*}
		
	\end{enumerate}
	
	{\large\textbf{Mô phỏng trong R}} \\
	\begin{lstlisting}[language=R]
		X <- function(){
			khanang <- c(1, 0)
			thosan1 <- sample(khanang, 1, prob = c(0.2, 0.8))
			thosan2 <- sample(khanang, 1, prob = c(0.4, 0.6))
			thosan3 <- sample(khanang, 1, prob = c(0.6, 0.4))
		
			return (thosan1 + thosan2 + thosan3)
		}
		
		meanX <- function(N){
			kq <- replicate(N, X())
			return (mean(kq))
		}
		
		#Test
		meanX(50000)
		#> meanX(50000)
		#[1] 1.19892	
	\end{lstlisting}
	
	\section{Câu 2}
	(3 đ). Một đồng xu có xác suất ra ngửa là 0.4. Gieo đồng xu đến khi ra ngửa thì dừng. Tính
	kì vọng của số lần gieo. \\
	
	\begin{center}
		\textbf{Giải}
	\end{center}
	
	Xét thí nghiệm tung đồng xu đến khi có mặt ngửa thì dừng.\\
	
	Xác suất tung được mặt ngửa: $p = 0.4$\\
	
	Gọi X là bnn "Số lần tung được mặt sấp" thì X có phân phối hình học (Geometric distribution):
	$$X \sim NB(1, 0.4)$$
	
	Do đó, kì vọng của số lần gieo đồng xu:
	$$E(X + 1) = E(X) + 1 = \frac{1 - 0.4}{0.4} + 1 = 2.5$$
	{\large\textbf{Mô phỏng trong R}} \\
	\begin{lstlisting}[language=R]
		X <- function(){
			khanag <- c(1, 0) #1: Head, 0: tail
			count <- 0
			while(TRUE){
				tung_dong_xu <- sample(khanag, 1, prob = c(0.4,0.6))
				count <- count + 1
				if(tung_dong_xu == 1){
					break
				}
			}
			return (count)
		}
		
		meanX <- function(N){
			kq <- replicate(N, X())
			return (mean(kq))
		}
		
		meanX(50000)
		#> meanX(50000)
		#[1] 2.5062
	\end{lstlisting}
	
	\section{Câu 3}
	(6 đ). Chọn ngẫu nhiên một số thực L trên đoạn [0, 1], dựng hình vuông có cạnh dài L mét. Tìm kì vọng của diện tích hình vuông \\
	
	\begin{center}
		\textbf{Giải}
	\end{center}
		
	Ta nhận thấy $X \sim Uniform(0, 1)$. Do đó, hàm mật độ xác suất của X là:
	\begin{equation*}
	f_X(x) = 
	\begin{cases}
	1, & \text{$0 \leq x \leq 1$} \\
	0, & \text{otherwise}
	\end{cases}
	\end{equation*}
	
	Gọi Y là diện tích của hình vuông với cạnh X(m).
	$$Y = X^2$$
		
	Ta thấy rằng Y là biến ngẫu nhiên phái sinh của biến ngẫu nhiên X qua hàm:
	$$r(x) = x^2, 0 \leq x \leq 1$$
	
	Theo công thức kì vọng, ta có:
	\begin{align*}
		E(Y) = E(r(X)) &= \int_{-\infty}^{\infty} r(x)f_X(x)dx \\
		&= \int_0^1 x^2dx \\
		&= \frac{1}{3} \approx 0.333
	\end{align*}	
	
	
	{\large\textbf{Mô phỏng trong R}} \\
	\begin{lstlisting}[language=R]
		Y <- function(){
			l <- runif(1, min = 0, max = 1)
			return (l^2)
		}

		meanY <- function(N){
			kq <- replicate(N, Y())
			return (mean(kq))
		}

		#Test
		meanY(50000)
		#> meanY(50000)
		#[1] 0.333004
	\end{lstlisting}
	
	\section{Tham khảo}
	\label{Tham khao}
	[1] Introduction to R, \href{https://campus.datacamp.com/courses/free-introduction-to-r}{Datacamp}. \\
	
\end{document}